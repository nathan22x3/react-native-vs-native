In a React Native app, its JavaScript code runs in a dedicated thread, while the rest of the app runs in what we’ll
call ``the native realm''.
JavaScript controls the application's business logic, while the native realm renders the UI and oversees device
interactions.
To communicate, these two realms rely on something known as ``the Bridge''.

The JavaScript thread and the native realm cannot communicate directly - they are unable to listen to, reply to, or
cancel events and actions occurring on the other side.
Instead, they use asynchronous message queues to send serialized messages back and forth.
The gap is ``bridged'' using this system.

Some performance concerns may develop as a result of the detached and asynchronous nature of this mode of
communication.
Queues can become congested if the user is rapidly scrolling through a big and complex list, for example — numerous
``user has scrolled'' and ``render this new UI'' updates fly back and forth.
Animations can likewise be a source of concern for the same reason.
In actuality, most of the time, these types of performance flaws are insignificant to the user.
