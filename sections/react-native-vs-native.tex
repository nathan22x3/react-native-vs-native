Both approaches to mobile development have significant drawbacks.
It is critical to consider when choosing between React Native and Native mobile app development.

\subsection{Development Time}

The primary difference between Native and React Native is that the code is written separately for both Android and iOS
platforms in the former approach due to technical differences.
You must assign two separate teams, namely the Android development team and the iOS development team, in order to make
your app available on both platforms.
Furthermore, whenever you make a change to the code, you must build and run the entire project.

In the case of React Native, the same developers write a single code for both platforms, which means you can build
React Native apps for both Android and iOS using the same code.
Furthermore, the concept of hot reloading makes it easier to deploy the project without having to rebuild each time,
reducing the effort and time required for React Native application development.

As such, if you want to enter the mobile market sooner and at a lower cost while targeting both platforms, use
\textbf{React Native}.

\subsection{Mobile App Cost}

As per the top mobile app development companies, app development with React Native can reduce the cost by 30\%-35\%
when compared to that of Native app development for the same app project.

So, if cost is your primary consideration, React Native is the right choice for you.

\subsection{UI/UX Experience}

Even though the React Native framework allows developers to reuse functionality code while writing platform-specific
code separately, developing a complex user interface such as custom views, navigation patterns, seamless transitions
and animations, and so on is difficult with React Native app development.
Furthermore, matching the UI/UX standards of both the Android and iOS platforms is difficult.

In contrast, in native application development, each screen is designed separately for both Android and iOS devices,
resulting in a better mobile app UI/UX experience.

Though, in terms of exceptional mobile app experience, the latter is the winner of React Native vs Native app
development.

\subsection{Scalability}

When compared to Native technology, React Native gives mobile app developers more options for processing more work in
an application and launching higher-functionality updates.
So, in terms of scalability, React Native takes the win over Native app development.

\subsection{Performance}

React Native uses JavaScript, which has a single dedicated device thread.
It is unable to carry out multiple asynchronous tasks at the same time.
Besides, the framework does not support many popular modules and functionalities.
As a result, it is impossible to use native device features, cutting-edge technologies, or perform complex
manipulations.

Native apps, on the other hand, are created using Swift, Java, or Kotlin, which are far superior to JavaScript for
working on advanced features, performing heavy calculations, and integrating advanced hardware devices.
This enables mobile app developers to create any type of application using Native technologies with ease.

To summarize, Native outperforms React Native in app performance.

\subsection{Native Apps Interactivity}

Interactivity with native apps is yet another important factor in understanding the difference between react native
and native approaches.

To provide an exceptional experience, a native mobile application can easily interact with other native apps and
access their data.
However, this is not possible with React Native.
React Native apps communicate with other native apps using third-party libraries, which limits data access.
When your app requires continuous interaction with other native mobile apps, React Native becomes a secondary option
for mobile app development.

\subsection{Support for APIs and Third-Party Libraries}

When it comes to API accessibility, Native app development frameworks can directly use all types of APIs. In the
case of React Native technology, however, this is not possible.

Only a few APIs can be used in the development of React Native apps.
To implement complex APIs, you must first create a connection layer using Native technologies.
When it comes to adding a plethora of APIs, this clearly shows that Native is a better choice than React Native.

\subsection{Native Module Support}

React Native can easily handle a wide range of cross-platform use cases, but it lacks the APIs required to cover all
native mobile features.
To access those inaccessible APIs, React Native relies on the concept of adding native module support, which
necessitates that the React Native app development company you hire is fluent in both native languages.

Nevertheless, there is no such constraint when developing a Native mobile app for Android or iOS devices, indicating
that Native app development is once again a better option in the native apps vs react native battle.

\subsection{App Security}

React Native framework is built on JavaScript, which is not a strongly typed, OOPs-based language like native
app development languages such as Java, Kotlin, Objective-C, and Swift.
Additionally, in the case of React Native, several third-party libraries and APIs are used, making it difficult to
identify errors and loopholes in the development process.
As a result, React Native is less secure than native app development technologies.
